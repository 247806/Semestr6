\documentclass{article}
\usepackage[T1]{fontenc}
\usepackage{hyperref}
\usepackage{amsmath}
\usepackage[utf8]{inputenc}
\usepackage[polish]{babel}
%{Informatyka stosowana 2020, I st., semestr VI}


\author{
	{Dominik Gałkowski, 247659} \\
	{Jan Śladowski, 247806}\\ 
{Prowadzący: dr inż. Marcin Kacprowicz}
}

\title{Komputerowe systemy rozpoznawania 2024/2025\\Projekt 1. Klasyfikacja dokumentów tekstowych}
\begin{document}
\maketitle


\section{Cel projektu}
\indent Celem projektu jest przygotowanie aplikacji, która będzie dokonywała klasyfikacji zbioru dokumentów tekstowych metodą k-NN. Jej zadaniem będzie przydzielenie obiektu do odpowiedniej klasy. W trakcie działania programu konieczne będzie dokonanie ekstrakcji wektorów cech z artykułów dostępnych pod linkiem: 
\url{https://archive.ics.uci.edu/dataset/137/reuters+21578+text+categorization+collection}. \\

\section{Klasyfikacja nadzorowana metodą $k$-NN.  Ekstrakcja cech, wektory cech}
Metoda k-NN (k-Nearest Neighbors) jest algorytmem leniwym, co oznacza, że nie tworzy wewnętrznej reprezentacji danych uczących, tylko przechowuje wszystkie wzorce uczące. Dopiero po pojawieniu się wzorca testowego, dla którego wyznaczana jest odległość względem wszystkich wzorców uczących, algorytm poszukuje rozwiązania. \cite{knn}. Algorytm k-NN wymaga dwóch kluczowych parametrów, metryki, za pomocą, której wyznacza odległości obiektu testującego od wszystkich wzorców uczących oraz liczby sąsiadów k, czyli elementów do których badany element ma najbliżej. Decyzja klasyfikacyjna opiera się na najczęstszej klasie wśród k najbliższych sąsiadów. W przypadku naszego projektu odległość pomiędzy obiektami oznacza skalę podobieństwa tekstów.

W projekcie ekstrakcja cech charakterystycznych tekstu jest dokonywana poprzez stworzenie wektora cech, opisanego na podstawie następujących cech:
\begin{enumerate}
    \item Długość tekstu - cecha ta oznacza liczbę słów, z których składa się dany artykuł, co pozwala na porównanie długości różnych tekstów.
        \begin{equation}
            len = \sum_{i=0}^{n} x_i
        \end{equation}
        gdzie \( x_i \) = liczba liter \( \geq 3 \).
    \item Dominująca waluta - cecha ta reprezentowana jest poprzez nazwę waluty, ze zbioru walut kluczowych, która pojawia się najczęściej w badanym artykule. Na przykład w przypadku, gdy w badanym tekście pojawi się dwukrotnie słowo "U.S. Dollar" i tylko raz "Japanese Yen" to dla tej cechy zostanie zwrócona wartość tekstowa "U.S. Dollar".
        \begin{equation}
            w = \arg\max_{w \in W} f(w)
        \end{equation}
        gdzie W - zbiór walut kluczowych, \( f(w) \)  - liczba wystąpień waluty \( w \) w tekście.
    \item Nazwy miejsca - cecha ta jest reprezentacją tekstową wszystkich miejsc, np. nazw miast lub regionów pojawiających się ze zbioru miejsc kluczowych. Przykład: "AMR Corp will hold a press conference this morning in New York at 0900 EST, a company spokesman said." wynikiem dla tego cytatu będzie zbiór \( M' = \{ \text{New York} \} \). 
        \begin{equation}
            M' =  x \in M  \land x \in T
        \end{equation}
        gdzie M - zbiór miejsc kluczowych, T - zbiór słów znajdujących się w tekście
    \item Liczba unikalnych słów - cecha oznaczająca wystąpienia słów unikalnych, czyli takich, które nie pojawiąją się więcej niż jeden raz w tekście. Przykład: "AMR Corp will hold a press conference this morning in New York at 0900 EST, a company spokesman said. And the next week also in New York", słowa "New York" nie zostaną zliczone.
        \begin{equation}
            uk = \mid x : x \in T \land f(x) = 1 \mid
        \end{equation}
        gdzie T - zbiór słów znajdujących się w tekście, \( f(x)\)  - funkcja zwracająca liczbę wystąpień słowa x w tekście.
    \item Średnia długość słowa - cecha opisująca średnią długość słów w badanym tekście.
     \begin{equation}
            al = \frac{\sum_{i=0}^{n} a_i}{\sum_{i=0}^{n} x_i}
        \end{equation}
        gdzie \( a_i \) - litera, \( x_i \) - liczba liter \( \geq 3 \).
    \item Liczba  słów kluczowych w pierwszych 3 zdaniach - cecha ta oznacza bezwględną liczbę wystąpień słów ze zbioru słów kluczowych w pewnym fragmencie tekstu (pierwsze 3 zdania).
        \begin{equation}
            fw = \ \mid x : x \in K \wedge x \in T_{\text{y}} \mid
        \end{equation}
        gdzie K - zbiór słów kluczowych, \( T_y \) - zbiór słów znajdujący się w pierwszych trzech zdaniach tekstu.
    \item Liczba słów zaczynających się wielką literą - cecha ta oznaczą liczbę wystąpień słów zaczynających się wielką literą, nie uzwlędniając przy tym słów rozpoczynających nowe zdanie.
        \begin{equation}
            bw = \sum_{i=0}^{n} x_i
        \end{equation}
        gdzie \( x_i \) = słowo zaczynające się wielką literą.
    \item Pierwsze kluczowe słowo w tekście - cecha opisująca pierwsze znalezione słowo znajdujące się w zbiorze słów kluczowych. Przykład: "AMR Corp will hold a press conference this morning in New York at 0900 EST, a company spokesman said." wynikiem dla tego cytatu będzie \(x_{first} \) = New York.
       \begin{equation}
            x_{first} = \min \{x: x \in K \land x \in T \}
        \end{equation}
        gdzie K - zbiór słów kluczowych, T - zbiór słów znajdujących się w tekście.
    \item Liczba słów kluczowych - cecha ta oznacza bezwzględną liczbę wystąpień słów ze zbioru słów kluczowych. 
        \begin{equation}
            kw = \ \mid x : x \in K \land x \in T \mid
        \end{equation}
        gdzie K - zbiór słów kluczowych, T - zbiór słów znajdujących się w tekście.
    \item Względna liczba słów kluczowych - cecha która reprezentuje stosunek słów kluczowych do długości całego tekstu. 
        \begin{equation}
            rw = \frac{ \mid x : x \in K \land x \in T \mid}{ \sum_{i=0}^{n} x_i}
        \end{equation}
       gdzie K - zbiór słów kluczowych, \( x_i \) = liczba liter \( \geq 3 \), T - zbiór słów znajdujących się w tekście.
    \item Nazwiska - cecha ta jest reprezentacją tekstową wszystkich nazwisk pojawiających się ze zbioru nazwisk kluczowych. Przykład: "Wallis was quoted as saying the Reagan Administration wants Japanese cooperation so the White House can ensure any U.S." wynikiem dla tego cytatu będzie zbiór \( N' = \{ \text{Reagan} \} \).
    \begin{equation}
            N' =  x \in N \land x \in T
        \end{equation}
        gdzie N - zbiór nazwisk kluczowych, T - zbiór słów znajdujących się w tekście.
\end{enumerate}

Wektor cech będzie miał postać: 
        \begin{equation}
          v = [c1, c2, c3, c4, c5, c6, c7, c8, c9, c10, c11]
        \end{equation}


\section{Miary jakości klasyfikacji} 
Miary jakości klasyfikacji (Accuracy, Precision,
Recall, F1). We wprowadzeniu zaprezentować minimum teorii potrzebnej do realizacji
zadania, tak by inżynier innej specjalności zrozumiał dalszy opis. Należy podać {\bf konkretne wzory miar użyte w tym eksperymencie oraz krótko
opisać ich znaczenie i zakresy przyjmowanych wartości. Należy podać przykładowe
wartości każdej miary. Nie przepisuj
teorii, ale podaj link/przypis i opisz jak rozumiesz jej zastosowanie w tym konkretnym
zadaniu}. \\
\indent Stosowane wzory, oznaczenia z objaśnieniami znaczenia symboli użytych w
doświadczeniu. Oznaczenia jednolite w obrębie całego sprawozdania.  Opis zawiera przypisy do bibliografii zgodnie z
Polską Normą, (zob. materiały BG PŁ).\\
\noindent {\bf Sekcja uzupełniona jako efekt zadania Tydzień 03 wg Harmonogramu Zajęć
na WIKAMP KSR.}


\section{Metryki i miary podobieństwa tekstów w klasyfikacji}
Wzory, znaczenia i opisy symboli zastosowanych metryk z
przykładami. Wzory, opisy i znaczenia miar
podobieństwa tekstów zastosowanych w obliczaniu metryk dla wektorów cech z
przykładami dla każdej miary \cite{niewiadomski08}.  Oznaczenia jednolite w obrębie całego sprawozdania.  {\bf Podaj metryki i miary
podobieństwa nie z literatury (te wystarczy zacytować linkiem), ale konkretne ich
postaci stosowane w zadaniu. Jakie zakresy wartości przyjmują te miary i
metryki, co oznaczają ich wartości? Podaj przykładowe wartości dla przykładowych wektorów cech}. \\ 
\noindent {\bf Sekcja uzupełniona jako efekt zadania Tydzień 04 wg Harmonogramu Zajęć
na WIKAMP KSR.}

\section{Wyniki klasyfikacji dla różnych parametrów wejściowych}
Wstępne wyniki miary Accuracy dla próbnych klasyfikacji na ograniczonym zbiorze tekstów (podać parametry i kryteria
wyboru wg punktów 3.-8. z opisu Projektu 1.). 
\noindent {\bf Sekcja uzupełniona jako efekt zadania Tydzień 05 wg Harmonogramu Zajęć
na WIKAMP KSR.}


\section{Dyskusja, wnioski, sprawozdanie końcowe}

Wyniki kolejnych eksperymentów wg punktów 2.-8. opisu projektu 1.  Każdorazowo
podane parametry, dla których przeprowadzana eksperyment. 
Wykresy (np. słupowe) i tabele wyników
obowiązkowe, dokładnie opisane w ,,captions'' (tytułach), konieczny opis osi i
jednostek wykresów oraz kolumn i wierszy tabel.\\ 

{**Ewentualne wyniki realizacji punktu 9. opisu Projektu 1., czyli ,,na ocenę 5.0'' i ich porównanie do wyników z
części obowiązkowej**.Dokładne interpretacje uzyskanych wyników w zależności od parametrów klasyfikacji
opisanych w punktach 3.-8 opisu Projektu 1. 
Omówić i wyjaśnić napotkane problemy (jeśli były). Każdy wniosek/problem powinien mieć poparcie
w przeprowadzonych eksperymentach (odwołania do konkretnych wyników: wykresów,
tabel). \\
\underline{Dla końcowej oceny jest to najważniejsza sekcja} sprawozdania, gdyż prezentuje poziom
zrozumienia rozwiązywanego problemu.\\

** Możliwości kontynuacji prac w obszarze systemów rozpoznawania, zwłaszcza w kontekście pracy inżynierskiej,
magisterskiej, naukowej, itp. **\\

\noindent {\bf Sekcja uzupełniona jako efekt zadań Tydzień 05 i Tydzień 06 wg Harmonogramu Zajęć
na WIKAMP KSR.}


\section{Braki w realizacji projektu 1.}
Wymienić wg opisu Projektu 1. wszystkie niezrealizowane obowiązkowe elementy projektu, ewentualnie
podać merytoryczne (ale nie czasowe) przyczyny tych braków. 


\begin{thebibliography}{0}
\bibitem{knn} https://home.agh.edu.pl/~horzyk/lectures/miw/KNN.pdf
\bibitem{niewiadomski08} A. Niewiadomski, Methods for the Linguistic Summarization of Data: Applications of Fuzzy Sets and Their Extensions, Akademicka Oficyna Wydawnicza EXIT, Warszawa, 2008.
\end{thebibliography}

Literatura zawiera wyłącznie źródła recenzowane i/lub o potwierdzonej wiarygodności,
możliwe do weryfikacji i cytowane w sprawozdaniu. 
\end{document}