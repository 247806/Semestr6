\documentclass{article}
\usepackage[a4paper, top=2cm, bottom=2cm, left=2cm, right=2cm]{geometry}
\usepackage[T1]{fontenc}
\usepackage[utf8]{inputenc}
\usepackage[polish]{babel}
\usepackage{amsmath}
\usepackage{url}
\usepackage{graphicx}
 \usepackage{float}
 \usepackage{pgfplots}
\pgfplotsset{compat=1.18}


\begin{document}
\section{Opisy atrybutów bazy danych}
\begin{enumerate}
    \item last\_updated - data przeprowadzenia pomiarów, z tego atrybutu zostanie wykorzystana godzina w celu określenia pory dnia, zakres [0 - 24]. 
    \item temperature\_celsius - temperatura wyrażona w stopniach Celsjusza w zakresie [-25, 50].
    \item wind\_kph - prędkość wiatru wyrażona w kilometrach na godzinę w zakresie [3, 81]. 
    \item pressure\_mb - ciśnienie powietrza wyrażone w milibarach w zakresie [964 - 1052]. 
    \item humidity - wilgotność w zakresie [2 - 100\%].
    \item visibility\_km - widoczność wyrażona w kilometrach w zakresie [0, 24].
    \item uv\_index - wartość promieniowania słonecznego UV w zakresie [0, 16].
    \item air\_quality\_Carbon\_Monoxide - pomiar jakości powietrza ze względu na stężenie tlenku węgla, wyrażony w ppm(liczba cząstek CO2 w milionie cząstek powietrza), w zakresie [0 - 3000].
    \item air\_quality\_Nitrogen\_dioxide - pomiar jakości powietrza ze względu na stężenie dwutlenku azotu, wyrażony w ppm(liczba cząstek NO2 w milionie cząstek powietrza) w zakresie [0 - 261].
    \item air\_quality\_gb-defra-index - skala określająca poziomy zanieczyszczenia w powietrzu w zakresie [1 - 10].

    
\end{enumerate}
\end{document}
