\documentclass{article}
\usepackage[T1]{fontenc}
\usepackage[utf8]{inputenc}
\usepackage[polish]{babel}
\usepackage{amsmath}
\usepackage{url}
\usepackage{graphicx}
 \usepackage{float}
 \usepackage{pgfplots}
\pgfplotsset{compat=1.18}
%{Informatyka stosowana 2022, I st., semestr VI}


\author{
	{Dominik Gałkowski, 247659} \\
	{Jan Śladowski, 247806}\\ 
{Prowadzący: dr inż. Marcin Kacprowicz}
}

\title{Komputerowe systemy rozpoznawania 2024/2025\\Projekt 2. Podsumowania lingwistyczne relacyjnych baz danych}
\begin{document}
\maketitle 

\section{Cel}
Celem projektu jest stworzenie aplikacji, której główną funkcjonalnością
jest lingwistyczna agregacja zawartości wybranego zbioru danych. Ma ona za zadanie generowanie podsumowań lingwistycznych dla wybranych przez użytkownika kwantyfikatorów, sumaryzatorów i kwalifikatorów dla różnych atrybutów. Analiza otrzymanych wyników polega na określeniu, znaczenia wybranych kwantyfikatorów, sumaryzatorów, kwalifikatorów oraz miar ich jakości dla wiarygodności i jakości otrzymanych podsumowań lingwistycznych. Przykładowe podsumowanie to, np. większość pomiarów ma wysokie ciśnienie.


\section{Baza danych, zmienne lingwistyczne, kwantyfikatory lingwistyczne}

\subsection{Charakterystyka podsumowywanej bazy danych}
W tym projekcie został wykorzystany zbiór danych zapisany w pliku w formacie .csv, na podstawie którego utworzono bazę danych - PostgresSQL. Baza danych o nazwie World Weather Repository zawiera różnego rodzaju pomiary danych atmosferycznych, np. temperatura lub prędkość wiatru. \cite{baza} Użteczność bazy jest określona na stronie Kaggle jako 10.0, a dane z niej są wykorzystywane do prognozowanie pogody oraz analizy klimatu na różnych kontynentach. Baza jest nabieżąco aktualizowana, natomiast na dzień 18.05.2025r. składa się z 70292 rekordów. 

Zmiennym lingwistycznym przypisuje się znaczenie ze względu na potrzebę lepszej interpretowalności danych przez użytkowników. Ludzie rzadko reagują na dokładne wartości (np. 1033.8 mb ciśnienia), natomiast określenie „wysokie ciśnienie” pozwala im intuicyjnie rozumieć sytuację pogodową. Stąd istnieje zapotrzebowanie na „przekładanie” danych formalnych na język naturalny. \\
Podmiotem podsumowań jest pomiar atmosferyczny, z bazy World Weather Repository wybrano 10 atrybutów, które zostaną rozmyte, są to następujące kolumny:
\begin{enumerate}
    \item last\_updated - data przeprowadzenia pomiarów, z tego atrybutu zostanie wykorzystana godzina w celu określenia pory dnia, zakres [0 - 24]. 
    \item temperature\_celsius - temperatura wyrażona w stopniach Celsjusza w zakresie [-25, 50].
    \item wind\_kph - prędkość wiatru wyrażona w kilometrach na godzinę w zakresie [3, 92]. 
    \item pressure\_mb - ciśnienie powietrza wyrażone w milibarach w zakresie [947 - 1052]. 
    \item humidity - wilgotność w zakresie [2 - 100\%].
    \item visibility\_km - widoczność wyrażona w kilometrach w zakresie [0, 32].
    \item uv\_index - wartość promieniowania słonecznego UVI w zakresie [0, 16].
    \item air\_quality\_Carbon\_Monoxide - pomiar jakości powietrza ze względu na stężenie tlenku węgla, wyrażony w ppm(liczba cząstek CO2 w milionie cząstek powietrza), w zakresie [0 - 3000].
    \item air\_quality\_Nitrogen\_dioxide - pomiar jakości powietrza ze względu na stężenie dwutlenku azotu, wyrażony w ppm(liczba cząstek NO2 w milionie cząstek powietrza) w zakresie [0 - 428].
    \item air\_quality\_gb-defra-index - skala określająca poziomy zanieczyszczenia w powietrzu w zakresie [1 - 10].

    
\end{enumerate}

\subsection{Zmienne lingwistyczne (atrybuty/własności obiektów)}
Poniżej zostały zaprezentowane zmienne lingwistyczne dla atrybutów opisanych w sekcji 2.1 oraz ich wzory analityczne. W każdym z poniższych wzorów \(L_x\) to zmienna ligwistyczna, $\mathcal{L}_x$ - nazwa zmiennej lingwistycznej , \(H_x\) zbiór możliwych przyjmowanych wartości, \(X_x\) - przestrzeń rozważań, \(x\) - numer kolejnej zmiennej ligwistycznej. 
\begin{enumerate}
    \item last\_updated
        \begin{equation}
            L_1 = \langle \mathcal{L}_1, H_1, \mathcal{X}_1 \rangle
        \end{equation}
        gdzie: $\mathcal{L}_1$ – pora dnia, $H_1$ – \{nocna, poranna, południowa, popołudniowa, wieczorna\}, $\mathcal{X}_1 = [0, 24]$. \\
        Poniżej wzory dla wszystkich możliwych etykiet.
        \begin{equation}
            \mu_{\text{nocna}}(x) =
            \begin{cases}
            \frac{7 - x}{3}, & x \in (4, 7) \\
            1, & x \in [0, 4] \\
            \frac{x - 21}{3}, & x \in [21, 24) \\
            0, & \text{w przeciwnym razie} \\
            \end{cases}
        \end{equation}

        \begin{equation}
            \mu_{\text{poranna}}(x) =
            \begin{cases}
            \frac{x - 4}{3}, & x \in (4, 7) \\
            1, & x \in [7, 9] \\
            \frac{12 - x}{3}, & x \in (9, 12) \\
            0, & \text{w przeciwnym razie} \\
            \end{cases}
        \end{equation}

        \begin{equation}
            \mu_{\text{południowa}}(x) =
            \begin{cases}
            \frac{x - 9}{3}, & x \in (9, 12) \\
            1, & x \in [12, 13] \\
            \frac{16 - x}{3}, & x \in (13, 16) \\
            0, & \text{w przeciwnym razie} \\
            \end{cases}
        \end{equation}

        \begin{equation}
            \mu_{\text{popołudniowa}}(x) =
            \begin{cases}
            \frac{x - 13}{3}, & x \in (13, 16) \\
            1, & x \in [16, 17] \\
            \frac{20 - x}{3}, & x \in (17, 20) \\
            0, & \text{w przeciwnym razie} \\
             \end{cases}
        \end{equation}

        \begin{equation}
            \mu_{\text{wieczorna}}(x) =
            \begin{cases}
            \frac{x - 17}{3}, & x \in (17, 20) \\
            1, & x \in [20, 21] \\
            \frac{24 - x}{3}, & x \in (21, 24) \\
            0, & \text{w przeciwnym razie} \\
            \end{cases}
        \end{equation} 
        
Wykresem funkcji przynależności znajduje w załączniku pod nazwą img/day.png.
    
    \item temperature\_celsius
        \begin{equation}
            L_2 = \langle \mathcal{L}_2, H_2, \mathcal{X}_2 \rangle
        \end{equation}
        gdzie: $\mathcal{L}_2$ – temperatura, $H_2$ – \{bardzo zimna, zimna, umiarkowana, ciepła, gorąca\}, $\mathcal{X}_2 = [-25, 50]$. \\
        Poniżej wzory dla wszystkich możliwych etykiet.
                \begin{equation}
                   \mu_{\text{bardzo\_zimna}}(x) =
                    \begin{cases}
                    1, & x \in [-25, -15] \\
                    \frac{-5 - x}{10}, & x \in (-15, -5) \\
                    0, & \text{w przeciwnym razie}
                    \end{cases}
                \end{equation}
                
                \begin{equation}
                   \mu_{\text{zimna}}(x) =
                    \begin{cases}
                    \frac{x + 10}{10}, & x \in (-15, -5) \\
                    1, & x \in [-5, 0] \\
                    \frac{10 - x}{10}, & x \in (0, 10) \\
                    0, & \text{w przeciwnym razie}
                    \end{cases}
                \end{equation}

                \begin{equation}
                    \mu_{\text{umiarkowana}}(x) =
                    \begin{cases}
                    \frac{x}{10}, & x \in (0, 10) \\
                    1, & x \in [10, 15] \\
                    \frac{20 - x}{10}, & x \in (15, 25) \\
                    0, & \text{w przeciwnym razie}
                    \end{cases}
                \end{equation}

                \begin{equation}
                    \mu_{\text{ciepła}}(x) =
                    \begin{cases}
                    \frac{x - 15}{10}, & x \in (15, 25) \\
                    1, & x \in [25, 30] \\
                    \frac{30 - x}{10}, & x \in (30, 40) \\
                    0, & \text{w przeciwnym razie}
                    \end{cases}
                \end{equation}

                \begin{equation}
                    \mu_{\text{gorąca}}(x) =
                    \begin{cases}
                    \frac{x - 30}{10}, & x \in (30, 40] \\
                    1, & x \in 40, 50] \\
                    0, & \text{w przeciwnym razie}
                    \end{cases}
                \end{equation}

Wykresem funkcji przynależności znajduje w załączniku pod nazwą img/temp.png.

    \item wind\_kph
        \begin{equation}
            L_3 = \langle \mathcal{L}_3, H_3, \mathcal{X}_3 \rangle
        \end{equation}
        gdzie: $\mathcal{L}_3$ – wiatr, $H_3$ – \{słaby, umiarkowany, silny, bardzo silny, gwałtowny\}, $\mathcal{X}_3 = [3, 151]$. \\
        Poniżej wzory dla wszystkich możliwych etykiet.
                  \begin{equation}
                    \mu_{\text{słaby}}(x) =
                    \begin{cases}
                    1, & x = 0 \\
                    \frac{20 - x}{20}, & x \in (0, 20) \\
                    0, & \text{w przeciwnym razie}
                    \end{cases}
                  \end{equation}
                \begin{equation}
                    \mu_{\text{umiarkowany}}(x) =
                    \begin{cases}
                    \frac{x}{20}, & x \in (0, 20) \\
                    1, & x = 20 \\
                    \frac{40 - x}{20}, & x \in (20, 40) \\
                    0, & \text{w przeciwnym razie}
                    \end{cases}
                  \end{equation}
                \begin{equation}
                    \mu_{\text{silny}}(x) =
                    \begin{cases}
                    \frac{x - 20}{20}, & x \in (20, 40) \\
                    1, & x = 40 \\
                    \frac{60 - x}{20}, & x \in (40, 60) \\
                    0, & \text{w przeciwnym razie}
                    \end{cases}
              \end{equation}
                \begin{equation}
                    \mu_{\text{bardzo\_silny}}(x) =
                    \begin{cases}
                    \frac{x - 40}{20}, & x \in (40, 60) \\
                    1, & x = 60 \\
                    \frac{80 - x}{20}, & x \in (60, 80) \\
                    0, & \text{w przeciwnym razie}
                    \end{cases}
              \end{equation}
              \begin{equation}
                    \mu_{\text{gwałtowny}}(x) =
                    \begin{cases}
                    \frac{x - 60}{20}, & x \in (60, 80) \\
                    1, & x \in [80, 92] \\
                    0, & \text{w przeciwnym razie}
                    \end{cases}
              \end{equation}

Wykresem funkcji przynależności znajduje w załączniku pod nazwą img/wind.png.
              
    \item pressure\_mb
        \begin{equation}
            L_4 = \langle \mathcal{L}_4, H_4, \mathcal{X}_4 \rangle
        \end{equation}
        gdzie: $\mathcal{L}_4$ – ciśnienie, $H_4$ – \{niskie, normalne, wysokie\}, $\mathcal{X}_4 = [947, 1050]$. \\
        Poniżej wzory dla wszystkich możliwych etykiet.
                \begin{equation}
                    \mu_{\text{niskie}}(x) =
                    \begin{cases}
                    1, & x \in [947, 970] \\
                    \frac{1000 - x}{30}, & x \in (970, 1000] \\
                    0, & \text{w przeciwnym razie}
                    \end{cases}
              \end{equation}
                \begin{equation}
                   \mu_{\text{normalne}}(x) =
                    \begin{cases}
                    \frac{x - 970}{30}, & x \in (970, 1000] \\
                    1, & x \in (1000, 1020] \\
                    \frac{1040 - x}{20}, & x \in (1020, 1040] \\
                    0, & \text{w przeciwnym razie}
                    \end{cases}
                \end{equation}

                \begin{equation}
                \mu_{\text{wysokie}}(x) =
                    \begin{cases}
                    \frac{x - 1020}{20}, & x \in (1020, 1040] \\
                    1, & x \in (1040, 1052] \\
                    0, & \text{w przeciwnym razie}
                    \end{cases}
                \end{equation}

Wykresem funkcji przynależności znajduje w załączniku pod nazwą img/pressure.png.
    
    \item humidity
    \begin{equation}
            L_5 = \langle \mathcal{L}_5, H_5, \mathcal{X}_5 \rangle
        \end{equation}
        gdzie: $\mathcal{L}_5$ – wilgotność powietrza, $H_5$ – \{suche, umiarkowane, wilgotne\}, $\mathcal{X}_1 = [2, 100]$. \\
        Poniżej wzory dla wszystkich możliwych etykiet.

        \begin{equation}
        \mu_{\text{sucho}}(x) =
        \begin{cases}
        1, & x \in (0, 10] \\
        \frac{50 - x}{40}, & x \in (10, 50) \\
        0, & \text{w przeciwnym razie}
        \end{cases}
        \end{equation}

        \begin{equation}
        \mu_{\text{umiarkowane}}(x) =
        \begin{cases}
        \frac{x - 10}{40}, & x \in (10, 50) \\
        1, & x = 50 \\
        \frac{80 - x}{30}, & x \in (50, 80) \\
        0, & \text{w przeciwnym razie}
        \end{cases}
        \end{equation}

        \begin{equation}
        \mu_{\text{wilgotne}}(x) =
        \begin{cases}
        \frac{x - 50}{30}, & x \in (50, 80) \\
        1, & x \in [80, 100] \\
        0, & \text{w przeciwnym razie}
        \end{cases}
        \end{equation}

Wykresem funkcji przynależności znajduje w załączniku pod nazwą img/humidity.png.

    \item visibility\_km
    \begin{equation}
            L_6 = \langle \mathcal{L}_6, H_6, \mathcal{X}_6 \rangle
        \end{equation}
        gdzie: $\mathcal{L}_6$ – stopień widoczności, $H_6$ – \{słaba, umiarkowana, dobra, bardzo dobra\}, $\mathcal{X}_6 = [0, 32]$. \\
        Poniżej wzory dla wszystkich możliwych etykiet.
        
    \begin{equation}
    \mu_{\text{słaba}}(x) =
    \begin{cases}
    1, & x \in [0, 4] \\
    \frac{8 - x}{4}, & x \in (4, 8] \\
    0, & \text{w przeciwnym razie}
    \end{cases}
    \end{equation}
    
    \begin{equation}
    \mu_{\text{umiarkowana}}(x) =
    \begin{cases}
    \frac{x - 4}{4}, & x \in (4, 8] \\
    1, & x \in (8, 12] \\
    \frac{16 - x}{4}, & x \in (12, 16] \\
    0, & \text{w przeciwnym razie}
    \end{cases}
    \end{equation}
    
    \begin{equation}
    \mu_{\text{dobra}}(x) =
    \begin{cases}
    \frac{x - 12}{4}, & x \in (12, 16] \\
    1, & x \in (16, 24] \\
    \frac{28 - x}{4}, & x \in (24, 28] \\
    0, & \text{w przeciwnym razie}
    \end{cases}
    \end{equation}

    \begin{equation}
    \mu_{\text{bardzo\_dobra}}(x) =
    \begin{cases}
    \frac{x - 24}{4}, & x \in (24, 28] \\
    1, & x \in (28, 32] \\
    0, & \text{w przeciwnym razie}
    \end{cases}
    \end{equation}

Wykresem funkcji przynależności znajduje w załączniku pod nazwą img/visibility.png.

    \item uv\_index
    \begin{equation}
            L_7 = \langle \mathcal{L}_7, H_7, \mathcal{X}_7 \rangle
        \end{equation}
        gdzie: $\mathcal{L}_7$ – promieniowanie UV, $H_1$ – \{niskie, umiarkowane, wysokie, bardzo wysokie, ekstremalne\}, $\mathcal{X}_7 = [0, 16]$. \\
        Poniżej wzory dla wszystkich możliwych etykiet.

    \begin{equation}
    \mu_{\text{niskie}}(x) =
    \begin{cases}
    1, & x \in [0, 2] \\
    \frac{3 - x}{1}, & x \in (2, 3] \\
    0, & \text{w przeciwnym razie}
    \end{cases}
    \end{equation}
    
    \begin{equation}
    \mu_{\text{umiarkowane}}(x) =
    \begin{cases}
    \frac{x - 2}{1}, & x \in (2, 3] \\
    1, & x \in (3, 5] \\
    \frac{6 - x}{1}, & x \in (5, 6) \\
    0, & \text{w przeciwnym razie}
    \end{cases}
    \end{equation}
    
    \begin{equation}
    \mu_{\text{wysokie}}(x) =
    \begin{cases}
    \frac{x - 5}{1}, & x \in (5, 6] \\
    1, & x \in (6, 7] \\
    \frac{8 - x}{1}, & x \in (7, 8) \\
    0, & \text{w przeciwnym razie}
    \end{cases}
    \end{equation}
    
    \begin{equation}
    \mu_{\text{bardzo\_wysokie}}(x) =
    \begin{cases}
    \frac{x - 7}{1}, & x \in (7, 8] \\
    1, & x \in (8, 10] \\
    \frac{11 - x}{1}, & x \in (10, 11) \\
    0, & \text{w przeciwnym razie}
    \end{cases}
    \end{equation}

    \begin{equation}
    \mu_{\text{ekstremalne}}(x) =
    \begin{cases}
    \frac{x - 10}{1}, & x \in (10, 11] \\
    1, & x \in (11, 16] \\
    0, & \text{w przeciwnym razie}
    \end{cases}
    \end{equation}

Wykresem funkcji przynależności znajduje w załączniku pod nazwą img/uv.png.

    \item air\_quality\_Carbon\_Monoxide
            \begin{equation}
            L_8 = \langle \mathcal{L}_8, H_8, \mathcal{X}_8 \rangle
        \end{equation}
        gdzie: $\mathcal{L}_8$ – zanieczyszczenie CO2, $H_8$ – \{normalne, niezdrowe, niebezpieczne\}, $\mathcal{X}_8 = [0, 2220]$. \\
        Poniżej wzory dla wszystkich możliwych etykiet.

    \begin{equation}
    \mu_{\text{normalne}}(x) =
    \begin{cases}
    1, & x \in [0, 200] \\
    \frac{500 - x}{300}, & x \in (200, 500) \\
    0, & \text{w przeciwnym razie}
    \end{cases}
    \end{equation}

    \begin{equation}
    \mu_{\text{wysokie}}(x) =
    \begin{cases}
    \frac{x - 200}{300}, & x \in (200, 500) \\
    1, & x \in [500, 800] \\
    \frac{900 - x}{300}, & x \in (800, 1100) \\
    0, & \text{w przeciwnym razie}
    \end{cases}
    \end{equation}
    
    \begin{equation}
    \mu_{\text{niezdrowe}}(x) =
    \begin{cases}
    \frac{x - 800}{300}, & x \in (800, 1100) \\
    1, & x \in [1100, 1700] \\
    \frac{2000 - x}{300}, & x \in (1700, 2000) \\
    0, & \text{w przeciwnym razie}
    \end{cases}
    \end{equation}
    
    \begin{equation}
    \mu_{\text{niebezpieczne}}(x) =
    \begin{cases}
    \frac{x - 1700}{300}, & x \in (1700, 2000) \\
    1, & x \in [2000, 3000] \\
    0, & \text{w przeciwnym razie}
    \end{cases}
    \end{equation}

Wykresem funkcji przynależności znajduje w załączniku pod nazwą img/co.png.
    
    \item air\_quality\_Nitrogen\_dioxide
                \begin{equation}
            L_9 = \langle \mathcal{L}_9, H_9, \mathcal{X}_9 \rangle
        \end{equation}
        gdzie: $\mathcal{L}_9$ – zanieczyszczenie NO2, $H_9$ – \{normalne, niezdrowe, niebezpieczne\}, $\mathcal{X}_9 = [0, 428]$. \\
        Poniżej wzory dla wszystkich możliwych etykiet.

    \begin{equation}
    \mu_{\text{normalne}}(x) =
    \begin{cases}
    1, & x \in [0, 50] \\
    \frac{100 - x}{50}, & x \in (50, 100] \\
    0, & \text{w przeciwnym razie}
    \end{cases}
    \end{equation}
    
    \begin{equation}
    \mu_{\text{niezdrowe}}(x) =
    \begin{cases}
    \frac{x - 50}{50}, & x \in (50, 100] \\
    1, & x \in (100, 200] \\
    \frac{250 - x}{50}, & x \in (200, 250) \\
    0, & \text{w przeciwnym razie}
    \end{cases}
    \end{equation}
    
    \begin{equation}
    \mu_{\text{niebezpieczne}}(x) =
    \begin{cases}
    \frac{x - 200}{50}, & x \in (200, 250] \\
    1, & x \in (250, 428] \\
    0, & \text{w przeciwnym razie}
    \end{cases}
    \end{equation}

Wykresem funkcji przynależności znajduje w załączniku pod nazwą img/no.png.
        
    \item Indeks jakości powietrza
                \begin{equation}
            L_10 = \langle \mathcal{L}_{10}, H_{10}, \mathcal{X}_{10} \rangle
        \end{equation}
        gdzie: $\mathcal{L}_{10}$ – jakość powietrza, $H_{10}$ – \{bardzo dobra, dobra, umiarkowana, zła, bardzo zła\}, $\mathcal{X}_{10} = [1, 10]$. \\
        Poniżej wzory dla wszystkich możliwych etykiet.

\begin{equation}
\mu_{\text{bardzo\_dobra}}(x) =
\begin{cases}
1, & x \in [1, 2] \\
\frac{3 - x}{1}, & x \in (2, 3] \\
0, & \text{w przeciwnym razie}
\end{cases}
\end{equation}

\begin{equation}
\mu_{\text{dobra}}(x) =
\begin{cases}
\frac{x - 2}{1}, & x \in (2, 3] \\
1, & x \in (3, 4] \\
\frac{5 - x}{1}, & x \in (4, 5) \\
0, & \text{w przeciwnym razie}
\end{cases}
\end{equation}

\begin{equation}
\mu_{\text{umiarkowana}}(x) =
\begin{cases}
\frac{x - 4}{1}, & x \in (4, 5] \\
1, & x \in (5, 6] \\
\frac{7 - x}{1}, & x \in (6, 7) \\
0, & \text{w przeciwnym razie}
\end{cases}
\end{equation}  

                \begin{equation}
                    \mu_{\text{zła}}(x) =
                    \begin{cases}
                    \frac{x - 6}{1}, & x \in (6, 7] \\
                    1, & x \in (7, 8] \\
                    \frac{9 - x}{1}, & x \in (8, 9)\\
                    0, & \text{w przeciwnym razie} \\
                    \end{cases}
                \end{equation}

                \begin{equation}
                    \mu_{\text{bardzo zła}}(x) =
                    \begin{cases}
                    \frac{x - 8}{1}, &  x \in (8, 9] \\
                    1, & x \in (9, 10] \\
                    0, & \text{w przeciwnym razie} \\
                    \end{cases}
                \end{equation}
Wykresem funkcji przynależności znajduje w załączniku pod nazwą img/air.png.
\end{enumerate}

\subsection{Kwantyfikatory lingwistyczne (liczności obiektów)}
Poniżej zostały zaprezentowane kwantyfikatory lingwistyczne wraz z ich wzorami analitycznymi. Wykresem funkcji przynależności znajduje w załączniku pod nazwą img/a.png.
\begin{itemize}
    \item[-] prawie żaden
        \begin{equation}
            \mu_{\text{prawie\_żaden}}(x) =
        \begin{cases}
        \exp\left( -\frac{1}{2} \left( \frac{x - 0{,}0}{0,06} \right)^2 \right), & x \in (0, 0.2) \\
        0, & \text{w przeciwnym razie}
        \end{cases}
        \end{equation}
    \item[-] trochę
        \begin{equation}
            \mu_{\text{trochę}}(x) =
            \begin{cases}
            \frac{x - 0.10}{0.15}, & x \in (0.10, 0.25) \\
            1, & x \in [0.25, 0.30] \\
            \frac{0.45 - x}{0.15}, & x \in (0.30, 0.45)\\
            0, & \text{w przeciwnym razie} \\
            \end{cases}
        \end{equation}
    \item[-] około połowa
        \begin{equation}
        \mu_{\text{około połowa}}(x) =
        \begin{cases}
        \exp\left( -\frac{1}{2} \left( \frac{x - 0{,}5}{0,06} \right)^2 \right), & x \in (0.35, 0.75) \\
        0, & \text{w przeciwnym razie}
\end{cases}
        \end{equation}
    \item[-] wiele
        \begin{equation}
             \mu_{\text{trochę}}(x) =
            \begin{cases}
            \frac{x - 0.60}{0.15}, & x \in (0.60, 0.75) \\
            1, & x \in [0.75, 0.80] \\
            \frac{0.95 - x}{0.15}, & x \in (0.80, 0.95)\\
            0, & \text{w przeciwnym razie} \\
            \end{cases}
        \end{equation}
    \item[-] prawie wszystkie
        \begin{equation}
            \mu_{\text{prawie wszystkie}}(x) =
            \begin{cases}
            \exp\left( -\frac{1}{2} \left( \frac{x - 1{,}0}{0,06} \right)^2 \right), &  x \in (0.8, 1.0) \\
            0, & \text{w przeciwnym razie}
            \end{cases}
        \end{equation}
\end{itemize}


\section{Narzędzia obliczeniowe: wybór/implementacja. Diagram UML klas do obliczeń rozmytych i generowania podsumowań}
Program został napisany w języku Java w wersji JDK 24. Do obsługi relacyjnej bazy danych PostgreSQL wykorzystano framework Spring Boot, który umożliwia wygodną konfigurację połączenia z bazą danych. Kod źródłowy komponentu odpowiedzialnego za obliczenia rozmyte został umieszczony w pakiecie o nazwie \texttt{fuzzy}. Znajduje sie w nim podstawowy model reprezentujący zbiory rozmyte, którego centralnym elementem jest abstrakcyjna klasa \texttt{FuzzySet}, która definiuje fundamentalne operacje na zbiorach rozmytych. Klasa ta udostępnia metody pozwalające na obliczanie podstawowych właściwości zbioru rozmytego, m.in. takich jak kardynalność oraz nośnik. Na bazie klasy \texttt{FuzzySet} zbudowane zostały trzy klasy reprezentujące konkretne typy funkcji przynależności: \texttt{GaussianFunction}, \texttt{TrapezoidalFunction} oraz \texttt{TriangularFunction}. Każda z tych klas implementuje matematyczną funkcję przynależności o charakterystycznym kształcie: odpowiednio gaussowskim, trapezoidalnym i trójkątnym. Dzięki temu możliwe jest precyzyjne modelowanie różnych pojęć językowych. \\
Oprócz modelu zbiorów rozmytych, pakiet \texttt{fuzzy} zawiera również klasy odpowiedzialne za tworzenie i przetwarzanie podsumowań lingwistycznych. Klasa \texttt{LinguisticVariable} reprezentuje zmienną lingwistyczną, która zawiera nazwę, przestrzeń rozważań oraz zestaw powiązanych z nią terminów lingwistycznych, np. „bardzo zimno”, „ciepło”, „gorąco”, które są reprezentowane przez odpowiednie zbiory rozmyte.
W strukturze podsumowań lingwistycznych wyróżniamy także kwalifikatory i sumaryzatory, które mają bardzo podobną strukturę, ponieważ są złożone z nazwy oraz przypisanego im zbioru rozmytego. Ich różnice wynikają jedynie z zastosowania, w jaki sposób są wykorzystywane w podsumowaniach lingwistycznych. W celu zachowania przejrzystości i uniknięcia duplikacji kodu, stworzona została wspólna klasa bazowa \texttt{LinguisticTerm}. 
Klasa \texttt{Quantifier} odpowiada za reprezentację kwantyfikatorów, takich jak „około połowy” czy „prawie żaden”. Kwantyfikator opisany jest za pomocą nazwy, funkcji przynależności oraz typu — może być kwantyfikatorem absolutnym lub względnym.
Ostatnim elementem pakietu jest klasa \texttt{Summary}. To ona odpowiada za generowanie podsumowań lingwistycznych oraz obliczanie miar jakości na podstawie przekazanych obiektów: sumaryzatora, kwalifikatora oraz kwantyfikatora. \\
Strukturę logiczną opisanych klas oraz zależności między nimi przedstawiono na poniższym diagramie UML.


\begin{figure}[H]
\centering
\includegraphics[width=\textwidth]{img/lingustic.png}
\caption{Diagram UML pakietu \texttt{fuzzy}.}
\end{figure}


\section{ Jednopodmiotowe podsumowania lingwistyczne. Miary jakości, podsumowanie optymalne}
Poniżej zaprezentowano wyniki eksperymentów polegających na generowaniu jednopodmiotowych podsumowań lingwistycznych. Eksperymenty zostały podzielone na trzy części. W każdym eksperymencie wykorzystano tylko kwalifikator relatywny. Dla każdego utworzonego podsumowania lingwistycznego obliczane były miary jakości oraz miara podsumowania optymalnego. W każdej z poniższych sekcji w tabelach przedstawiono wybrane podsumowania, dla których wartość miary \textit{degree of truth} (\(T_1\)) była większa niż 0{,}1. \\
W tabelach zaprezentowano treść podsumowania oraz wartość miary \(T_1\). Pozostałe miary jakości (\(T_2\)–\(T_{11}\)) zostały załączone w pliku \textit{Załącznik1.pdf}. Dodatkowo dla wybranego przykładu w każdej sekcji przedstawiono również pełny zestaw miar jakości.


\subsection{Podsumowania lingwistyczne pierwszego typu}
W tej sekcji zostały przedstawione wyniki dla podsumowań lingwistycznych z jednym sumaryzatorem.
W celu obliczenia miary podsumowania optymalnego przyjęto następujące wagi: \(w1 = 0.23\), \(w2 = 0.11\), \(w3 = 0.11\), \(w4 = 0.11\), \(w5 = 0.11\), \(w6 = 0.11\), \(w7 = 0.11\), \(w8 = 0.11\), \(w9 = 0.00\), \(w_{10} = 0.00\), \(w_{11} = 0.00\). Dla miar jakości \(T_9\), \(T_{10}\) oraz \(T_{11}\) przyjęto wagi o wartościach 0.00, ponieważ w tych podsuwowaniach nie korzystano z kwalifikatora.

\begin{table}[H]
    \centering
    \begin{tabular}{|c|c|c|}
    \hline
    \textbf{Lp.} &\textbf{Podsumowanie} & \textbf{T1}  \\ \hline
    1. & Prawie żaden pomiarów  jest/ma zimną temperaturę & 0.58 \\ \hline
    2. & Około połowy pomiarów  jest/ma ciepła temperaturę & 0.22 \\ \hline
    3. & Trochę pomiarów  jest/ma popołudniową godzinę & 0.85 \\ \hline
    4. & Prawie żaden pomiarów jest/ma wieczorną godzinę & 0.36 \\ \hline
    5. & Około połowy pomiarów jest/ma wilgotne powietrze & 0.85 \\ \hline
    6. & Trochę pomiarów  jest/ma umiarkowane powietrze & 0.77 \\ \hline
    7. & Prawie żaden pomiarów jest/ma silny wiatr & 0.57\\ \hline
    8. & Około połowy pomiarów jest/ma słaby wiatr & 0.25 \\ \hline
    9. & Prawie wszystkie pomiarów jest/ma normalne cieśnienie & 0.73 \\ \hline
    10. & Prawie żaden pomiarów  jest/ma wysokie cisnienie & 0.77 \\ \hline
    11. & Prawie wszystkie pomiarów jest/ma umiarkowana widoczność & 0.35 \\ \hline
    12. & Prawie żaden pomiarów jest/ma dobra widoczność & 0.93 \\ \hline
    13. & Trochę pomiarów  jest/ma bardzo wysokie promieniowe UV & 0.23 \\ \hline
    14. & Prawie żaden pomiarów  jest/ma umiarkowane promieniowanie UV & 0.02 \\ \hline
    15. & Około połowy pomiarów  jest/ma normalne zanieczyszczenie CO2 & 0.34 \\ \hline
    16. & Prawie żaden pomiarów  jest/ma niebezpieczne zanieczyszczenie CO2 & 0.95 \\ \hline 
    17. & Prawie wszystkie pomiarów jest/ma normalne zanieczyszczenie NO2 & 0.52 \\ \hline
    18. & Prawie żaden pomiarów  jest/ma niezdrowe zanieczyszczenie NO2 & 0.81 \\ \hline
    19. & Trochę pomiarów jest/ma dobra jakość powietrza & 0.40 \\ \hline
    20. & Prawie żaden pomiarów  jest/ma bardzo złą jakość powietrza & 0.59 \\ \hline

    \end{tabular}
    \caption{Wyniki jednopodmiotowych podsumowań lingwistycznych typu 1 z \textit{degree of truth}.}
\end{table}

W poniższej tabeli przedstawiono miary jakości oraz miarę podsumowania optymalnego dla podsumowania lingwistycznego nr.1 z tabeli nr.1. Zdanie: \textit{„Prawie żaden pomiarów jest/ma zimną temperaturę”} zostało ocenione na podstawie następujących wskaźników: \(T_1\) – degree of truth (0{,}58) oznacza, że podsumowanie jest zgodne z danymi w~58\%. \(T_2\) – degree of imprecision (0{,}88) wskazuje na dużą nieprecyzyjność sumaryzatora wynikającą z szerokiego nośnika funkcji przynależności. \(T_3\) – degree of covering (0{,}12) oznacza, że tylko niewielka część danych spełnia warunki zawarte w podsumowaniu. \(T_4\) – degree of appropriateness (0{,}00) przyjmuje wartość zerową ze względu na jeden sumaryzator. \(T_5\) – length of summary (1{,}0) wskazuje, że podsumowanie zawiera jeden sumaryzator. \(T_6\) – degree of quantifier imprecision (0{,}80) oznacza wysoką nieprecyzyjność kwantyfikatora „prawie żaden”, wynikającą z szerokiego obszaru przynależności. \(T_7\) – degree of quantifier cardinality (0{,}60) oznacza umiarkowaną liczebność kwantyfikatora. \(T_8\) – degree of summarizer imprecision (0{,}94) wskazuje na bardzo wysoką nieprecyzyjność summarizera „zimna temperatura”. \(T_9\), \(T_{10}\) i \(T_{11}\) przyjmują wartość 0{,}00, co wynika z braku kwalifikatora w podsumowaniu. Ostateczna miara podsumowania optymalnego (\(T\)) wynosi 0{,}61 i wskazuje na umiarkowanie dobrą jakość wygenerowanego podsumowania.


  \begin{table}[H]
    \centering
    \begin{tabular}{|c|c|c|c|c|c|c|c|c|c|c|c|}
    \hline
    \textbf{\(T_1\)} &\textbf{\(T_2\)} & \textbf{\(T_3\)} & \textbf{\(T_4\)} & \textbf{\(T_5\)} & \textbf{\(T_6\)} & \textbf{\(T_7\)} & \textbf{\(T_8\)} & \textbf{\(T_9\)} & \textbf{\(T_{10}\)} & \textbf{\(T_{11}\)} & \textbf{\(T\)} \\ \hline
    0.58 & 0.88 & 0.12 & 0.00 & 1.00 & 0.80 & 0.60 & 0.94 & 0.00 & 0.00 & 0.00 & 0.61 \\ \hline
    \end{tabular}
    \caption{Miary jakości dla podsumowania nr. 1 z tabeli nr.1.}
\end{table}  

\subsection{Podsumowania lingwistyczne pierwszego typu z dwoma sumaryzatorami}
W tej sekcji zostały przedstawione wyniki dla podsumowań lingwistycznych z dwoma sumaryzatorami.
W celu obliczenia miary podsumowania optymalnego przyjęto następujące wagi: \(w1 = 0.23\), \(w2 = 0.11\), \(w3 = 0.11\), \(w4 = 0.11\), \(w5 = 0.11\), \(w6 = 0.11\), \(w7 = 0.11\), \(w8 = 0.11\), \(w9 = 0.00\), \(w2 = 0.00\), \(w2 = 0.00\). Dla miar jakości \(T_9\), \(T_{10}\) oraz \(T_{11}\) przyjęto wagi o wartościach 0.00, ponieważ w tych podsumowaniach nie korzystano z kwalifikatora.

\begin{table}[H]
\begin{center}
\normalsize % lub \Large, \normalsize, itp.
\begin{tabular}{|c|p{10cm}|c|} % p{10cm} = zawijanie tekstu, szerokość dostosuj
\hline
\textbf{Lp.} & \textbf{Podsumowanie} & \textbf{T1} \\ \hline
1. & Trochę pomiarów jest/ma ciepła temperaturę i jest/ma popołudniowa godzinę & 0.41 \\ \hline
2. & Prawie żaden pomiarów jest/ma ciepła temperaturę i jest/ma umiarkowane UV & 0.38 \\ \hline
3. & Około połowy pomiarów jest/ma normalne ciśnienie i jest/ma umiarkowany wiatr & 0.91 \\ \hline
4. & Trochę pomiarów jest/ma normalne ciśnienie i jest/ma popołudniowa godzinę & 0.83 \\ \hline
5. & Około połowy pomiarów jest/ma normalne ciśnienie i jest/ma ciepła temperaturę & 0.24 \\ \hline
6. & Trochę pomiarów jest/ma popołudniowa godzinę i jest/ma umiarkowany wiatr & 0.41 \\ \hline
7. & Trochę pomiarów jest/ma popołudniowa godzinę i jest/ma umiarkowaną wilgotność & 0.19 \\ \hline
8. & Prawie żaden pomiarów jest/ma wieczorną godzinę i jest/ma dobrą widoczność & 0.99 \\ \hline
9. & Trochę pomiarów jest/ma popołudniową godzinę i jest/ma normalne zanieczyszczenie NO2 & 0.79 \\ \hline
10. & Prawie żaden pomiarów jest/ma niezdrowe zanieczyszczenie NO2 i jest/ma bardzo złą jakość powietrza & 0.96 \\ \hline
\end{tabular}
\caption{Wyniki jednopodmiotowych podsumowań lingwistycznych typu 1 z dwoma sumaryzatorami z miarą \textit{degree of truth}.}
\end{center}
\end{table}

W poniższej tabeli przedstawiono miary jakości oraz miarę podsumowania optymalnego dla podsumowania lingwistycznego nr.3 z tabeli nr.3. Zdanie: \textit{„Około połowy pomiarów jest/ma normalne ciśnienie i jest/ma umiarkowany wiatr”} uzyskało bardzo wysoką wartość \(T_1\) – (0{,}91), co oznacza, że podsumowanie to jest w~91\% zgodne z danymi. \(T_2\) – (0{,}01) wskazuje na bardzo niski poziom nieprecyzyjności sumaryzatorów, co świadczy o ich dobrej definicji. \(T_3\) – (0{,}99), znaczna część danych spełnia warunki podsumowania. \(T_4\) – przyjmuje wartość bardzo niską (\(1{,}60 \times 10^{-5}\)). \(T_5\) –  (0{,}5) wskazuje, że w podsumowaniu użyto dwóch sumaryzatorów. \(T_6\) – degree of quantifier imprecision (0{,}60) sugeruje umiarkowaną trudność w precyzyjnym zdefiniowaniu kwantyfikatora „około połowy”. \(T_7\) – (0{,}62) oznacza, że użyty kwantyfikator nie jest szczególnie precyzyjny pod względem liczebności. \(T_8\) - (0{,}28) wskazuje na stosunkowo niski poziom nieprecyzyjności sumaryzatorów. Wartości \(T_9\), \(T_{10}\) i \(T_{11}\) są równe 0{,}00, ponieważ podsumowanie nie zawiera kwalifikatora. Ostateczna miara jakości podsumowania (\(T\)) wynosi 0{,}54, co sugeruje umiarkowaną jakość podsumowania, mimo jego wysokiej zgodności z danymi.


\begin{table}[H]
    \centering
    \begin{tabular}{|c|c|c|c|c|c|c|c|c|c|c|c|}
    \hline
    \textbf{\(T_1\)} &\textbf{\(T_2\)} & \textbf{\(T_3\)} & \textbf{\(T_4\)} & \textbf{\(T_5\)} & \textbf{\(T_6\)} & \textbf{\(T_7\)} & \textbf{\(T_8\)} & \textbf{\(T_9\)} & \textbf{\(T_{10}\)} & \textbf{\(T_{11}\)} & \textbf{\(T\)} \\ \hline
    0.91 & 0.01 & 0.99 & \(1.60 \times 10^{-5}\) & 0.5 & 0.60 & 0.62 & 0.28 & 0.00 & 0.00 & 0.00 & 0.54 \\ \hline
    \end{tabular}
    \caption{Miary jakości dla podsumowania nr. 3 z tabeli nr.3.}
\end{table}  


\subsection{Podsumowania lingwistyczne drugiego typu z jednym sumaryzatorem}
W tej sekcji zostały przedstawione wyniki dla podsumowań lingwistycznych z jednym sumaryzatorem i jednym kwalifikatorem.
W celu obliczenia miary podsumowania optymalnego przyjęto następujące wagi: \(w1 = 0.20\), \(w2 = 0.08\), \(w3 = 0.08\), \(w4 = 0.08\), \(w5 = 0.08\), \(w6 = 0.08\), \(w7 = 0.08\), \(w8 = 0.08\), \(w9 = 0.08\), \(w2 = 0.08\), \(w2 = 0.08\).

\begin{table}[H]
\begin{center}
\normalsize % lub \Large, \normalsize, itp.
\begin{tabular}{|c|p{10cm}|c|} % p{10cm} = zawijanie tekstu, szerokość dostosuj
\hline
\textbf{Lp.} & \textbf{Podsumowanie} & \textbf{T1} \\ \hline
1. & Wiele pomiarów będący/mający wysokie ciśnienie jest/ma południowa godzine & 1.0 \\\hline
2. & Trochę pomiarów będący/mający wysokie ciśnienie jest/ma popołudniowa godzine & 0.72 \\ \hline
3. & Prawie żaden pomiarów będący/mający wysokie ciśnienie jest/ma wieczorna godzine & 0.70 \\ \hline
4. & Około połowy pomiarów będący/mający gwałtowny waitr jest/ma południowa godzinę & 0.94 \\ \hline
5. & Wiele pomiarów będący/mający gorąca temperaturę jest/ma popołudniowa godzinę & 0.55 \\ \hline
6. & Około połowy pomiarów będący/mający suche powietrze jest/ma popołudniowa godzinę & 0.99 \\\hline
7. & Wiele pomiarów będący/mający silny wiatr jest/ma ciepła temperaturę & 0.84 \\ \hline
8. & Prawie wszystkie pomiarów będący/mający wysokie UV jest/ma ciepła temperaturę & 0.16 \\ \hline
9. & Prawie żaden pomiarów będący/mający normalne zanieczyszczenie CO2 jest/ma gorąca temperaturę & 0.58 \\ \hline
10. & Około połowy pomiarów będący/mający zła jakość powietrza jest/ma ciepła temperaturę & 0.89 \\ \hline
11. & Prawie wszystkie pomiarów będący/mający normalne zanieczysczenie CO2 jest/ma normalne zanieczysczenie NO2 & 0.74 \\ \hline
12. & Prawie żaden pomiarów będący/mający bardzo wysokie UV jest/ma niebezpieczne zanieczyszczenie NO2 & 0.99 \\ \hline
13. & Trochę pomiarów będący/mający normalne cisnienie jest/ma umiarkowane wilgotność & 0.72 \\ \hline
14. & Trochę pomiarów będący/mający gwałtowny wiatr jest/ma słaba widoczność & 0.86 \\ \hline
15. & Prawie żaden pomiarów będący/mający dobra widoczność jest/ma niebezpieczne zanieczysczenie CO2 & 0.91 \\ \hline

\end{tabular}
\caption{Wyniki jednopodmiotowych podsumowań lingwistycznych typu 2 z miarą \textit{degree of truth}.}
\end{center}
\end{table}

W poniższej tabeli przedstawiono miary jakości oraz miarę podsumowania optymalnego dla podsumowania lingwistycznego nr.15 z tabeli nr.5. Zdanie: \textit{„Prawie żaden pomiarów jest/ma niezdrowe zanieczyszczenie NO\textsubscript{2} i jest/ma bardzo złą jakość powietrza”} uzyskało bardzo wysoką wartość \(T_1\) – (0{,}91), co świadczy o wysokiej zgodności podsumowania. \(T_2\)  – (0{,}98) wskazuje na wysoki poziom nieprecyzyjności sumaryzatorów. \(T_3\)  – (0{,}03) oznacza, że tylko niewielka część danych spełnia warunki podsumowania. \(T_4\)  – (0{,}01). \(T_5\)  – (1{,}00) wskazuje na obecność jednego sumaryzatora. \(T_6\)  – (0{,}80) świadczy o trudności w precyzyjnym określeniu kwantyfikatora „prawie żaden”. \(T_7\)  – (0{,}60) wskazuje na średnią wartość liczebności kwantyfikatora. \(T_8\)  – (0{,}98) potwierdza dużą nieprecyzyjność sumaryzatora. Wartości \(T_9\)  (0{,}97) oraz \(T_{10}\)  – (0{,}97) wskazują na znaczną nieprecyzyjność i rozmycie kwalifikatora. \(T_{11}\)  – 1{,}00 oznacza, że wystepuje jeden kwalifikator. Miara \(T\)  – optymalna jakość podsumowania – wynosi 0{,}77, co wskazuje na stosunkowo wysoką jakość końcową.


\begin{table}[H]
    \centering
    \begin{tabular}{|c|c|c|c|c|c|c|c|c|c|c|c|}
    \hline
    \textbf{\(T_1\)} &\textbf{\(T_2\)} & \textbf{\(T_3\)} & \textbf{\(T_4\)} & \textbf{\(T_5\)} & \textbf{\(T_6\)} & \textbf{\(T_7\)} & \textbf{\(T_8\)} & \textbf{\(T_9\)} & \textbf{\(T_{10}\)} & \textbf{\(T_{11}\)} & \textbf{\(T\)} \\ \hline
    0.91 & 0.98 & 0.03 & 0.01 & 1.00 & 0.80 & 0.60 & 0.98 & 0.97 & 0.97 & 1.00 & 0.77 \\ \hline
    \end{tabular}
    \caption{Miary jakości dla podsumowania nr. 15 z tabeli nr.5.}
\end{table}  


\section{Wielopodmiotowe podsumowania lingwistyczne i~ich miary jakości} 
Wyniki kolejnych eksperymentów wg punktów 2.-4. opisu projektu 2. Uzasadnienie i
metoda podziału zbioru danych na rozłączne podmioty. Listy podsumowań
wielopodmiotowych i tabele/rankingi podsumowań dla danych atrybutów obowiązkowe i
dokładnie opisane w ,,captions'' (tytułach), konieczny opis kolumn i wierszy tabel.
{\bf Wzorów podsumowań ani miar nie należy przepisywać ani cytować, wystarczy podać literaturę, ale
należy skomentować co oznaczają i jaką informacje niosą wybrane miary w wybranych
przypadkach.} Konieczne uwzględnienie wszystkich 4-ch form podsumowań wielopodmiotowych. 
\\ 

** Możliwe sformułowanie zagadnienia wielopodmiotowego podsumowania optymalnego **.\\
\indent {** Ewentualne wyniki realizacji punktu ,,na ocenę 5.0'' wg opisu Projektu 2. i ich porównanie do wyników z
części obowiązkowej **.}\\

\noindent {\bf Sekcja uzupełniona jako efekt zadania Tydzień 12 wg Harmonogramu Zajęć
na WIKAMP KSR.}


\section{Dyskusja, wnioski}
Dokładne interpretacje uzyskanych wyników w zależności od parametrów klasyfikacji
opisanych w punktach 3.-4 opisu Projektu 2. 
Omówić i wyjaśnić napotkane problemy (jeśli były). Każdy wniosek/problem powinien mieć poparcie
w przeprowadzonych eksperymentach (odwołania do konkretnych wyników: tabel i miar
jakości). Ocena które podsumowania i dlaczego niosą najistotniejsze informacje
i które ich miary jakości mają małe albo duże znaczenie dla wiarygodności i jakości otrzymanych
agregacji/podsumowań.  \\
\underline{Dla końcowej oceny jest to najważniejsza sekcja} sprawozdania, gdyż prezentuje poziom
zrozumienia rozwiązywanego problemu.\\

** Możliwości kontynuacji prac w obszarze logiki rozmytej i wnioskowania rozmytego, zwłaszcza w kontekście pracy inżynierskiej,
magisterskiej, naukowej, itp. **\\

\noindent {\bf Sekcja uzupełniona jako efekt zadań Tydzień 11 i Tydzień 12 wg
Harmonogramu Zajęć na WIKAMP KSR.}


\section{Braki w realizacji projektu 2.}
Wymienić wg opisu Projektu 2. wszystkie niezrealizowane obowiązkowe elementy projektu, ewentualnie
podać merytoryczne (ale nie czasowe) przyczyny tych braków. 


\begin{thebibliography}{99}
\bibitem{baza} World Weather Repository - kaggle, \url{https://www.kaggle.com/datasets/nelgiriyewithana/global-weather-repository?resource=download}. [dostęp 18.05.2025r.]
 \bibitem{niewiadomski19} A. Niewiadomski, Zbiory rozmyte typu 2. Zastosowania w reprezentowaniu informacji.  Seria „Problemy współczesnej informatyki” pod redakcją L. Rutkowskiego. Akademicka Oficyna Wydawnicza EXIT, Warszawa, 2019.
\bibitem{zadrozny06} S. Zadrożny, Zapytania nieprecyzyjne i lingwistyczne podsumowania baz danych, EXIT, 2006, Warszawa
\bibitem{niewiadomski08} A. Niewiadomski, Methods for the Linguistic Summarization of Data: Applications of Fuzzy Sets and Their Extensions, Akademicka Oficyna Wydawnicza EXIT, Warszawa, 2008.
\end{thebibliography}

Literatura zawiera wyłącznie źródła recenzowane i/lub o potwierdzonej wiarygodności,
możliwe do weryfikacji i cytowane w sprawozdaniu. 
\end{document}
