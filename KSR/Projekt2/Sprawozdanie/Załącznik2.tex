\documentclass{article}
\usepackage[a4paper, top=2cm, bottom=2cm, left=2cm, right=2cm]{geometry}
\usepackage[T1]{fontenc}
\usepackage[utf8]{inputenc}
\usepackage[polish]{babel}
\usepackage{amsmath}
\usepackage{url}
\usepackage{graphicx}
 \usepackage{float}
 \usepackage{pgfplots}
\pgfplotsset{compat=1.18}


\begin{document}


\section{Kwantyfikatory lingwistyczne}
    \begin{itemize}
    \item[-] prawie żaden
        \begin{equation}
            \mu_{\text{prawie\_żaden}}(x) =
        \begin{cases}
        \exp\left( -\frac{1}{2} \left( \frac{x - 0{,}0}{0,06} \right)^2 \right), & x \in (0, 0.2) \\
        0, & \text{w przeciwnym razie}
        \end{cases}
        \end{equation}
    \item[-] trochę
        \begin{equation}
            \mu_{\text{trochę}}(x) =
            \begin{cases}
            \frac{x - 0.10}{0.15}, & x \in (0.10, 0.25) \\
            1, & x \in [0.25, 0.30] \\
            \frac{0.45 - x}{0.15}, & x \in (0.30, 0.45)\\
            0, & \text{w przeciwnym razie} \\
            \end{cases}
        \end{equation}
    \item[-] około połowa
        \begin{equation}
        \mu_{\text{około połowa}}(x) =
        \begin{cases}
        \exp\left( -\frac{1}{2} \left( \frac{x - 0{,}5}{0,06} \right)^2 \right), & x \in (0.35, 0.75) \\
        0, & \text{w przeciwnym razie}
\end{cases}
        \end{equation}
    \item[-] wiele
        \begin{equation}
             \mu_{\text{trochę}}(x) =
            \begin{cases}
            \frac{x - 0.60}{0.15}, & x \in (0.60, 0.75) \\
            1, & x \in [0.75, 0.80] \\
            \frac{0.95 - x}{0.15}, & x \in (0.80, 0.95)\\
            0, & \text{w przeciwnym razie} \\
            \end{cases}
        \end{equation}
    \item[-] prawie wszystkie
        \begin{equation}
            \mu_{\text{prawie wszystkie}}(x) =
            \begin{cases}
            \exp\left( -\frac{1}{2} \left( \frac{x - 1{,}0}{0,06} \right)^2 \right), &  x \in (0.8, 1.0) \\
            0, & \text{w przeciwnym razie}
            \end{cases}
        \end{equation}
\end{itemize}
\end{document}
